\documentclass{article}
\usepackage[utf8]{inputenc}
\title{ザッパな勉強記録}
\author{yanyanmaru}
\date{\today}
\begin{document}
\maketitle
\section{統計検定2級レベル}
\subsection{1章}
【1】2次関数 $f(x)=2x^2-4x+3$ のグラフと直線 $y=x+1$ について,次の問に答えなさい。\\
(1) $f(x)$のグラフの頂点の座標を求めなさい。\\
(2) $f(x)$のグラフと直線$y=x+1$ を描き,交点の座標を求めて書き入れなさい。\\

\subsection{2章}
\subsection{3章}
\subsection{4章}
\subsection{5章}

\section{機械学習}
\subsection{機械学習って何?}
コンピューターがデータから学習して、パターンを学習させ、未知のデータを判断するモデルを獲得することである。

\subsection{教師あり学習と教師なし学習って何が違うの?}
\subsubsection{教師あり学習}
\textbf{教師あり学習}とは、すでに存在するデータの入力データと出力データをアルゴリズムにあらかじめ与えて、正解データを計算する機械学習方法のことである。 \\
例:メールがスパムかどうか自動判定する、投票率を予測する

\subsubsection{教師あり学習は2つに分けられる???}
教師あり学習は回帰と分類の2つに分けることができます。\\
\textbf{回帰}とは連続するデータの値の将来の値を予測することである。よくある機械学習やなあ。\\
\textbf{分類}とはあらかじめ定めた分類に振り分けることである。あんま想像できんなあ。

\subsubsection{教師あり学習:分類}


\end{document}